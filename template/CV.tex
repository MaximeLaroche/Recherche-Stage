

\documentclass[french,12pt]{resume} % Use the custom resume.cls style
\usepackage{xcolor}

\usepackage[left=0.75in,top=0.6in,right=0.75in,bottom=0.6in]{geometry} % Document margins
\newcommand{\tab}[1]{\hspace{.2667\textwidth}\rlap{#1}}
\newcommand{\itab}[1]{\hspace{0em}\rlap{#1}}
\name{Maxime Laroche} % Your name
\address{Montréal, Québec} 
\address{514 348-0345 \\ Maxime-1.Laroche@polymtl.ca} 

\begin{document}
	
	%
	
	\begin{rSection}{Éducation}
		
		{\bf Baccalauréat en génie informatique} \hfill {\em  2020 - 2023} 
		\\ Polytechnique Montréal.
		\\GPA: 3,4 
		
		
		{\bf Baccalauréat en génie électrique} \hfill {\em  2018 - 2019} 
		\\ Polytechnique Montréal.
		\\ Bourse d'excellence à l'admission.
		\\ Changé d'orientation en cours d'étude.
		
	\end{rSection}
	
	
	\begin{rSection}{Projets}

		 \begin{rSubsection}{Voiture autonome}{2021}{}{}
			\item Utilisation d'un microprocesseur ATmega324PA pour qu'une petite voiture puisse naviguer dans son environnement.
			\item La voiture analyse son environnement grâce à des capteurs ultrason et peut ainsi éviter des obstacles et changer de chemin.
			\item Gestion de l'équipe pour assurer une bonne collaboration et le respect des échéances.
			 
		 \end{rSubsection}


		\begin{rSubsection}{Serveur TCP de Fichier}{2021}{}{}
			\item Conception d'une application serveur client de style «Google Drive» avec java.
			\item Permets de faire des sauvegardes automatiques des appareils sur le réseau local, ainsi que de partager des fichiers entre les appareils du réseau.
		\end{rSubsection}
		
		\begin{rSubsection}{Système de contrôle médiatique analogique}{2019}{}{}
			\item   Transport de plusieurs signaux audio, ainsi que l'état d'interrupteurs à l'aide d'un seul fil pour transporter les données.
			\item Planifier les contraintes de chacune des parties pour assurer une compatibilité du travail de chaque membre de l'équipe.
			\item \item  Analyse des coûts.
			\item Organisation de l'équipe pour assurer un processus de conception fluide.
		\end{rSubsection}

		
		% \begin{rSubsection}{Conception et réalisation d’un jeu de roulette }{2018}{}{}
		% 	\item  Recherche de solutions en collaboration avec une équipe de cinq personnes pour la réalisation du jeu.
		% 	\item  Analyse des coûts.
		% 	\item Organisation de l'équipe pour assurer un processus de conception fluide.
		% \end{rSubsection} 

		
	\end{rSection}
	
	\begin{rSection}{Expérience}
		
		\begin{rSubsection}{Ingénierie de service}{2020}{Bombardier}{}{}
			\item Conception de logiciels de collection et analyse de données pour optimiser l'organisation du travail au sein de la compagnie.
			\item Projet fait en conception itérative. 
			\item Apprentissage d'un langage sur le tas.
			\item Conception d'équipement de diagnostic d'avions.
		\end{rSubsection}
		
		\begin{rSubsection}{Assistant-chercheur en géométrie nodale}{2018}{Collège Ahuntsic, Montréal}{}
			\item 	Conception et exécution d'expériences pour confirmer ou infirmer les théories élaborées par le chercheur.
			\item   Effectuer des comptes rendus au chercheur des résultats obtenus.
			\item   Conception de matériel à l’aide d’une imprimante 3D.
		\end{rSubsection}
		
%		\begin{rSubsection}{Tuteur}{Janvier 2017 - Mai 2018}{Collège Ahuntsic, Montréal}{}
			%\item   Chimie générale (Chimie 1 au collégial)
			%\item   Calcul avancé  (Calcul 3 au collégial ou Calcul 1 à l'université)
			%\item Tuteur de ces cours alors que je les suivais
			%    \subitem    Voir la matière au début de la semaine et aider un étudiant vers la fin de la semaine
			%    \subitem    Obligation de toujours être à jour
%			\item   Simplifier et vulgariser des connaissances techniques pour faciliter l'apprentissage de l'étudiant
			%\item Moyenne de l'étudiant à augmenté de 20\% pendant ce tutorat
%		\end{rSubsection}
		
	\end{rSection}
	
	

	
	\begin{rSection}{Connaissances générales}
		
		\begin{tabular}{ @{} >{\bfseries}l @{\hspace{6ex}} l }
			Programmation \              & C++, Python, Java, Matlab, Git, VBA \\
			Outils généraux                 & MS Office, Latex \\
			Langues & Français et Anglais
		\end{tabular}
		
	\end{rSection}
	
	

\end{document}


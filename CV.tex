\documentclass{resume} % Use the custom resume.cls style
\usepackage{xcolor}
\usepackage{hyperref}
\usepackage[left=0.75in,top=0.6in,right=0.75in,bottom=0.6in]{geometry} % Document margins
\newcommand{\tab}[1]{\hspace{.2667\textwidth}\rlap{#1}}
\newcommand{\itab}[1]{\hspace{0em}\rlap{#1}}
\name{Maxime Laroche} % Your name
\address{Montréal, Québec} % Your address
%\address{123 Pleasant Lane \\ City, State 12345} % Your secondary addess (optional)
\address{(514) 348 - 0345 \\ 	mlaroche2009@gmail.com \\ \href{https://maximelaroche.github.io/portfolio}{maximelaroche.github.io/portfolio}} % Your phone number and email

\begin{document}

%----------------------------------------------------------------------------------------
% EDUCATION SECTION
%----------------------------------------------------------------------------------------

\begin{rSection}{Éducation}

	{\bf Maîtrise en génie informatique} \hfill {\em 2023 - 2024}
	\\ Polytechnique Montréal
	\\ Accent sur l'intelligence artificielle

		{\bf Baccalauréat en génie logiciel} \hfill {\em 2020 - 2023}
	\\ Polytechnique Montréal
	\\GPA: 3,36 \hspace{0.5cm}  Bourse d'excellence reçue à l'admission


\end{rSection}

%--------------------------------------------------------------------------------
%    Projects And Seminars
%-----------------------------------------------------------------------------------------------
\begin{rSection}{Projets}
	\begin{rSubsection}{Analyse de sentiments sur la politique américaine}{2023}{}{}

		\item Collaboration avec une équipe de cinq pour implémenter un modèle pouvant classifier des opinions par parti politique
		\item Utilisation d'un transformer pour classifier le texte
		\item Développement d'un agrégateur tweets pour obtenir un sentiment général de l'opinions des utilisateurs américains
		\item Utilisation de régression linéaire avec le modèle pour déterminer les mot-clèfs à utiliser dans les requêtes de l'API twitter
		%\item \url{http://maximelaroche.gitlab.io/log2990-201/#/home}
	\end{rSubsection}
	\begin{rSubsection}{Traducteur automatique}{2022}{}{}
		\item Implémentation d'un transformer pour effectuer la traduction automatique de l'anglais vers des commandes sqarql
	\end{rSubsection}

\end{rSection}


%----------------------------------------------------------------------------------------
% WORK EXPERIENCE SECTION
%----------------------------------------------------------------------------------------

\begin{rSection}{Experience}
	\begin{rSubsection}{Ingénieur en apprentissage machine}{2023}{Morgan Stanley}{}{}
		\item Entrainer un LLM attribuer déterminer les politiques de sécurités applicables à un projet en fonction du contenu du document d'architechture
	\end{rSubsection}
	\begin{rSubsection}{Ingénieur full stack}{2022}{Morgan Stanley}{}{}
		\item Conception et implémentation d'un site web en utilisant React et spring boot
		\item Déploiement de bases de données avec liguibase
		\item Écrire des tests de régressions
	\end{rSubsection}
	\begin{rSubsection}{Ingénierie de service}{2020}{Bombardier}{}{}
		\item  Conception de logiciels de collection et analyse de données pour optimiser l'organisation du travail au sein de la compagnie
		\item Apprentissage d'un langage (VBA) sur le tas
		\item Projet fait en conception itérative
		\item Conception d'équipement de diagnostic d'avions
	\end{rSubsection}

\end{rSection}


%----------------------------------------------------------------------------------------
% TECHNICAL STRENGTHS SECTION
%----------------------------------------------------------------------------------------

\begin{rSection}{Connaissances générales}

	\begin{tabular}{ @{} >{\bfseries}l @{\hspace{6ex}} l }
		Langage de programmation \  & Python, Typescript, Java, C++ \\
		Langues                & Français et Anglais
	\end{tabular}

\end{rSection}

\end{document}


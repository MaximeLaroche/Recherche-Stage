\documentclass{resume} % Use the custom resume.cls style
\usepackage{xcolor}
\usepackage{hyperref}
\usepackage[left=0.75in,top=0.6in,right=0.75in,bottom=0.6in]{geometry} % Document margins
\newcommand{\tab}[1]{\hspace{.2667\textwidth}\rlap{#1}}
\newcommand{\itab}[1]{\hspace{0em}\rlap{#1}}
\name{Maxime Laroche} % Your name
\address{Montréal, Québec} % Your address
%\address{123 Pleasant Lane \\ City, State 12345} % Your secondary addess (optional)
\address{(514) 348 - 0345 \\ mlaroche2009@gmail.com} % Your phone number and email

\begin{document}

%----------------------------------------------------------------------------------------
% EDUCATION SECTION
%----------------------------------------------------------------------------------------

\begin{rSection}{Éducation}

	{\bf Maîtrise en génie informatique} \hfill {\em 2023 - 2024}
	\\ Polytechnique Montréal
	\\ Accent sur l'intelligence artificielle
	\\ GPA: 3,33

		{\bf Baccalauréat en génie logiciel} \hfill {\em 2020 - 2023}
	\\ Polytechnique Montréal
	\\GPA: 3,41 \hspace{0.5cm}  Bourse d'excellence reçue à l'admission


\end{rSection}

%--------------------------------------------------------------------------------
%    Projects And Seminars
%-----------------------------------------------------------------------------------------------
\begin{rSection}{Projets}
	\begin{rSubsection}{Analyse de sentiments sur la politique américaine}{2023}{}{}

		\item Collaboration avec une équipe de cinq pour implémenter un modèle pouvant classifier des opinions par parti politique
		\item Utilisation d'un transformer pour classifier le texte
		\item Développement d'un agrégateur tweets pour obtenir un sentiment général de l'opinion des utilisateurs américains
		\item Utilisation de régression linéaire avec le modèle pour déterminer les mots-clefs à utiliser dans les requêtes de l'API Twitter
	\end{rSubsection}
	\begin{rSubsection}{Traducteur automatique}{2022}{}{}
		\item Implémentation d'un transformer pour effectuer la traduction automatique de l'anglais vers des commandes Sqarql
	\end{rSubsection}

\end{rSection}


%----------------------------------------------------------------------------------------
% WORK EXPERIENCE SECTION
%----------------------------------------------------------------------------------------

\begin{rSection}{Experience}
	\begin{rSubsection}{Opérations Datahub}{2023-2024}{Banque Nationale du Canada}{}{}
		\item  Aide aux opérations pour les requêtes opérationnelles du Datahub
		\item Automatisation de traitement de requêtes opérationnelles
	\end{rSubsection}
	\begin{rSubsection}{Conception de modèles d'apprentissage machine}{2023}{Morgan Stanley}{}{}
		\item Concevoir et entrainer un LLM à déterminer les politiques de sécurités applicables à un projet en fonction du contenu du document d'architecture
	\end{rSubsection}
	\begin{rSubsection}{Développeur full stack}{2022}{Morgan Stanley}{}{}
		\item Conception et implémentation d'un site web en utilisant React et Spring Boot
		\item Déploiement de bases de données avec Liquibase
		\item Écrire des tests de régressions
	\end{rSubsection}

\end{rSection}


%----------------------------------------------------------------------------------------
% TECHNICAL STRENGTHS SECTION
%----------------------------------------------------------------------------------------

\begin{rSection}{Connaissances générales}

	\begin{tabular}{ @{} >{\bfseries}l @{\hspace{6ex}} l }
		Langage de programmation \  & Python, Typescript, Java, C++ \\
		Langues                & Français et Anglais
	\end{tabular}

\end{rSection}

\end{document}


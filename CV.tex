\documentclass{resume} % Use the custom resume.cls style
\usepackage{xcolor}
\usepackage{hyperref}

\usepackage[left=0.75in,top=0.6in,right=0.75in,bottom=0.6in]{geometry} % Document margins
\newcommand{\tab}[1]{\hspace{.2667\textwidth}\rlap{#1}}
\newcommand{\itab}[1]{\hspace{0em}\rlap{#1}}
\name{Maxime Laroche} % Your name
\address{Montréal, Québec} % Your address
%\address{123 Pleasant Lane \\ City, State 12345} % Your secondary addess (optional)
\address{(514) 348 - 0345 \\ 	mlaroche2009@gmail.com \\ \href{https://maximelaroche.github.io/portfolio}{maximelaroche.github.io/portfolio}} % Your phone number and email

\begin{document}
	
	%----------------------------------------------------------------------------------------
	% EDUCATION SECTION
	%----------------------------------------------------------------------------------------
	
	\begin{rSection}{Education}

		{\bf Masters of Software Engineering} \hfill {\em 2023 - 2024} 
		\\ Polytechnique Montréal
		\\ Focus on artificial intelligence and data science
		
		{\bf Bachelor of Software Engineering} \hfill {\em 2020 - 2023} 
		\\ Polytechnique Montréal
		\\GPA: 3,36 \hspace{0.5cm}  Excellence scholarship received at admission.
		
		
	\end{rSection}
	
	%--------------------------------------------------------------------------------
	%    Projects And Seminars
	%-----------------------------------------------------------------------------------------------
	\begin{rSection}{Projects}
		\begin{rSubsection}{Sentiment analysis on american politics}{2023}{}{}
			\item Collaboration with a team of five to implement a model that can classify opinions by political party
			\item Use of transformers to classify text
			\item Develop an aggregator to be able to get the overall opinion of American twitter users
			\item Use of linear regression to determine most important keywords
			%\item \url{http://maximelaroche.gitlab.io/log2990-201/#/home}
		 \end{rSubsection}
		\begin{rSubsection}{Automatic translator}{2022}{}{}
			\item Implementation of a transformer to perform automatic English to Sparql translation
		 \end{rSubsection}
		
	\end{rSection}
	
	
	%----------------------------------------------------------------------------------------
	% WORK EXPERIENCE SECTION
	%----------------------------------------------------------------------------------------
	
	\begin{rSection}{Experience}
		\begin{rSubsection}{Cybersecurity and data engineer}{May - August 2023}{Morgan Stanley}{}{}
			\item Train a Large language model to tag architecture documents written in English with relevant internal security policies
			\item Write security policies in OPA
		\end{rSubsection}
		\begin{rSubsection}{Full stack engineer}{May - August 2022}{Morgan Stanley}{}{}
			\item Design and implementation of a website + back end server using react and spring boot.
			\item Database deployments based on liquibase
			\item writing end to end and unit tests
		\end{rSubsection}
		\begin{rSubsection}{Service engineering}{January - August 2020}{Bombardier}{}{}
			\item Design of data collection and analyzing software made to optimize workflows within the company.
			\item Learning a new language (VBA) on the spot
			\item Project made with iterative design
			\item Ground support equipment design.
		\end{rSubsection}
		
	\end{rSection}
	
	
	%----------------------------------------------------------------------------------------
	% TECHNICAL STRENGTHS SECTION
	%----------------------------------------------------------------------------------------
	
	\begin{rSection}{General knowledge}
		
		\begin{tabular}{ @{} >{\bfseries}l @{\hspace{6ex}} l }
			Programming languages \              & Python, Java, C++\\
			Languages & French and English
		\end{tabular}
		
	\end{rSection}
	
\end{document}


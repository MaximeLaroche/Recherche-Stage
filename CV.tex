

\documentclass[french,12pt]{resume} % Use the custom resume.cls style
\usepackage{xcolor}

\usepackage[left=0.75in,top=0.6in,right=0.75in,bottom=0.6in]{geometry} % Document margins
%\newcommand{\tab}[1]{\hspace{.2667\textwidth}\rlap{#1}}
%\newcommand{\itab}[1]{\hspace{0em}\rlap{#1}}
\name{Maxime Laroche} % Your name
\address{Montréal, Québec} 
\address{514 348-0345 \\ Maxime-1.Laroche@polymtl.ca} 

\begin{document}
	\begin{rSection}{Éducation}
		
		{\bf Baccalauréat en génie logiciel} \hfill {\em  2020 - 2023} 
		\\ Polytechnique Montréal.
		\\GPA: 3,34 
		
		
		{\bf Baccalauréat en génie électrique} \hfill {\em  2018 - 2019} 
		\\ Polytechnique Montréal.
		\\ Bourse d'excellence à l'admission.
		\\ Changé d'orientation en cours d'étude.
		
	\end{rSection}
	
	
	\begin{rSection}{Projets}

		\begin{rSubsection}{Jeux scrabble en linge}{2021}{}{}
			\item   Collaboration dans une équipe de 6 pour faire un jeu de scrabble en ligne
			\item Développement agile
			\item Angular pour le client, SocketIo pour la synchronisation, MongoDB pour la persistance des données et express (NodeJS) pour le serveur
		\end{rSubsection}
		\begin{rSubsection}{Serveur TCP de Fichier}{2021}{}{}
			\item Conception d'une application serveur client de style «Google Drive» avec Java.
			\item Permets de faire des sauvegardes automatiques des appareils sur le réseau local, ainsi que de partager des fichiers entre les appareils du réseau.
		\end{rSubsection}
		 \begin{rSubsection}{Voiture autonome}{2021}{}{}
			\item Utilisation d'un microprocesseur ATmega324PA pour qu'une petite voiture puisse naviguer dans son environnement.
			\item La voiture analyse son environnement grâce à des capteurs ultrason et peut ainsi éviter des obstacles et changer de chemin.
			\item Projet réalisé avec Git et C++.
			\item Gestion de l'équipe pour assurer une bonne collaboration et le respect des échéances.
			 
		 \end{rSubsection}


		
	\end{rSection}
	
	\begin{rSection}{Expérience}
		
		\begin{rSubsection}{Ingénierie de service}{2020}{Bombardier}{}{}
			\item Conception de logiciels de collection et analyse de données pour optimiser l'organisation du travail au sein de la compagnie.
			\item Projet fait en conception itérative. 
			\item Apprentissage d'un langage sur le tas.
			\item Conception d'équipement de diagnostic d'avions.
		\end{rSubsection}
		
		\begin{rSubsection}{Assistant-chercheur en géométrie nodale}{2018}{Collège Ahuntsic, Montréal}{}
			\item 	Conception et exécution d'expériences pour confirmer ou infirmer les théories élaborées par le chercheur.
			\item   Effectuer des comptes rendus au chercheur des résultats obtenus.
			\item   Conception de matériel à l’aide d’une imprimante 3D.
		\end{rSubsection}
		
		
	\end{rSection}
	
	

	
	\begin{rSection}{Connaissances générales}
		
		\begin{tabular}{ @{} >{\bfseries}l @{\hspace{6ex}} l }
			Language de programmation \              & C++, Python, Java, Typescript \\
			Outils généraux                 & MS Office, Latex, SQL, Angular, NodeJS, git \\
			Langues & Français et Anglais
		\end{tabular}
		
	\end{rSection}
	
	

\end{document}


\documentclass[12pt,french]{letter}
\usepackage{babel}
\usepackage{hyperref}
\usepackage[a4paper,margin=1in]{geometry}

\begin{document}
% If you want headings on subsequent pages,
% remove the ``%'' on the next line:
\pagestyle{headings}

\begin{letter}{Avonni Creator}
\address{Montréal, Canada}

\opening{Chers recruteurs,}


Étant un étudiant motivé, qui approche à la fin du baccalauréat, j’aimerais appliquer pour un stage et ce, qui débuterait en mai.  Présentement à la Polytechnique de Montréal, faisant ma concentration en génie informatique, les cours sont directement liés à la programmation Web et à la gestion de projets.  Faire partie d’une compagnie innovatrice comme la vôtre fait partie de mes objectifs.

Voici quelques points clefs:

En 2021, mes collègues et moi avons travaillé sur un projet: c’était fort intéressant et d’ailleurs, j’ai accordé la majorité de mon temps dessus.  Cela m'a permis d’acquérir de l’expérience en développement full stack.  Ceci consistait à implémenter un jeux de Scrabble en ligne! Angular était le framework utilisé pour l'application client de ce projet. Il devrait alors être relativement facile pour moi de s'adapter à un environnement React puisque les deux frameworks sont similaires.  Tout construire un site Internet comme ça m’a fait réaliser à quel point travailler dans ce domaine serait passionnant.  Après tout, le monde digital est vraiment le futur, que ce soit construire des sites Internets, des publicités, des jeux, etcétéra!  Cela va de soi que j’aimerais bien vous offrir mes services de développeur.  Tout ce qui est en rapport à la technologie, l’innovation, et les choses futuristiques m’intéresse énormément! 

En 2020, un stage a été complété à Bombardier.  Malgré que la durée devait être de 4 mois seulement, on m’avait offert de le prolonger pour un autre 4 mois supplémentaires.  C’est à cet emploi où j’ai réalisé que les emplois en génie étaient exigeants, mais très intéressants. Exigeant, dans le sens qu’à l’université, travailler avec de l'aide est la norme, en quelque sorte, mais dans le marché du travail, il y a certaines fois où qu’on doive apprendre à se débrouiller sur certaines parties d’un projet seul.  C’était très demandant parfois, mais étant une personne motivée, débrouillarde et enjouée, les défis me donnaient juste plus de curiosité à chaque fois!

Ceci étant dit, l'ouverture de postes au sein de votre entreprise m’intéresse fortement.
L’idée de faire partie de cette équipe me donne beaucoup d’énergie!


\signature{Maxime Laroche}

\closing{Au plaisir,}

%enclosure listing
%\encl{aaa}

\end{letter}
\end{document}
